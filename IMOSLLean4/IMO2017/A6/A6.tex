Let $R$ be a ring, $S$ be an abelian (additive) group, and $\iota : S \to R$ be a group homomorphism.
Find all functions $f : R \to S$ such that for any $x, y \in R$,
\[ f(\iota(f(x)) \iota(f(y))) + f(x + y) = f(xy). \tag{*}\label{2017a6-eq0} \]

We say that $f : R \to S$ is \emph{$\iota$-good} if $f$ satisfies the above functional equation.
We say that $f$ is \emph{non-periodic $\iota$-good} if $f$ is $\iota$-good and $f$ has no non-zero period.
The main goal is to find $\id_R$-good functions, but working in generality allows better proofs.
To classify $\iota$-good functions, we reduce to non-periodic case.





\subsection*{Answer}

Let $I$ be a double-sided ideal of $R$ and $[\cdot] : R \to R/I$ be the projection map.
Let $\phi : R/I \to G$ be a group homomorphism such that $[\iota(\phi(x))] = x$ for all $x \in R/I$.
Let $a \in Z(R/I)$ such that $a^2 = 1$.
Then the function $f(x) = \phi(a(1 - [x]))$ is $\iota$-good.
This construction gives us all the $\iota$-good functions in the following cases:
\begin{enumerate}

    \item 
    $G$ is $2$-torsion free and $3$-torsion free.
    
    \item
    $G = R$ is a simple ring of characteristic not equal to $2$ and $\iota = \id_R$.
    Since $Z(R)$ is a field, this means that the good functions are $0$, $x \mapsto 1 - x$, and $x \mapsto x - 1$.
    
    \item
    $G = R$ is a field of characteristic $2$ and $\iota = \id_R$.
    That is, the good functions are $0$ and $x \mapsto x + 1$.

\end{enumerate}





\subsection*{References}

\begin{itemize}
    
    \item
    \url{https://artofproblemsolving.com/community/c6h1480146p8693244}.
    
    Solution in AoPS by \textbf{anantmudgal09} (post \#75).
    We follow the main step of this solution when $G$ is $2$-torsion-free.

    \item
    \url{https://artofproblemsolving.com/community/c6h1480146p29214012}.

    Solution in AoPS by \textbf{BlazingMuddy} (author of this project, post \#176).
    We follow the main step of this solution when $R$ is a field.
    More generally, the steps allows computing $f(x)$ when $x$ is a unit of $R$.

\end{itemize}





\subsection*{Excellent functions}

Let $R$ be a ring and $G$ be an abelian group.
An \emph{excellent function} from $R$ to $G$ is a function $f : R \to G$ such that for all $x, y \in R$,
\[ f(x + y - xy) + f(1 - (x + y)) = f(1 - xy). \tag{**}\label{2017a6-eq-excellent0} \]

Excellent functions appear in the classification of non-periodic good functions.
Namely, an injective non-periodic good function induces an excellent function.
Clearly, group homomorphisms are excellent functions.
The main question of interest is determining if the converse holds.
In this part, we try to answer this question and explore some properties of excellent function.

By plugging $y = 0$ into~\eqref{2017a6-eq-excellent0}, we get $f(x) + f(1 - x) = f(1)$ for all $x \in R$.
By plugging $y = 1$ instead, we get $f(1) + f(-x) = f(1 - x)$, which implies $f(x) + f(1) = f(x + 1)$ for all $x \in R$.
Thus we also get $f(-x) = -f(x)$ for all $x \in R$ and $f(0) = 0$.
By induction on $n \in \Z$, we get $f(n) = n f(1)$ and $f(x + n) = f(x) + n f(1)$ for all $x \in R$.
By replacing $(x, y)$ with $(-x, -y)$ and using the equalities $f(-x) = -f(x)$ and $f(x + 1) = f(x) + 1$, one gets
\[ f(xy + x + y) = f(xy) + f(x + y). \tag{3.1}\label{2017a6-eq-excellent1} \]

Now we continue with more equations.
The above one yields
\[ f((x + 1)(y + 1)) = f(xy) + f(x + y) + f(1). \]
By induction on $n \in \Z$, we get that for any $x, y \in R$,
\[ f((x + n)(y + n)) = f(xy) + n f(x + y) + n^2 f(1). \tag{3.2}\label{2017a6-eq-excellent2} \]
Plugging $y = 0$ and replacing $x$ with $x - n$, we get $f(nx) = n f(x)$ for any $n \in \Z$ and $x \in R$.
We now prove:

\begin{lemma}\label{2017a6-excellent-main}
For any $a, b \in R$, we have $2 f(3a + b) = 2 (3 f(a) + f(b))$.
In particular, $f$ must be a group homomorphism if $G$ is $2$-torsion free and $3$-torsion free.
\end{lemma}
\begin{proof}
By replacing $x$ with $2x$ and using $n = 2$ in~\eqref{2017a6-eq-excellent2}, we get that for any $x, y \in R$,
\[ 2 f((x + 1)(y + 2)) = 2 (f(xy) + f(2x + y) + 2 f(1)). \]
By symmetry, for any $x, y \in R$,
\[ 2 f((x + 2)(y + 1)) = 2 (f(xy) + f(x + 2y) + 2 f(1)). \]
We can now write $2 f((x + 3)(y + 3))$ in two ways:
\[ 2(f(xy) + 3 f(x + y) + 9 f(1)) = 2 f((x + 3)(y + 3)) = 2 (f(xy) + f(2x + y) + f(x + 2y) + 9 f(1)). \]
After some cancellations, we simplify to
\[ 6 f(x + y) = 2 (f(2x + y) + f(x + 2y)). \]
The lemma follows by change of variables $(x, y) = (2a + b, -(a + b))$ and slight algebraic manipulation.
\end{proof}



\textbf{Some further directions.}
The set of excellent functions, say $E(R, G)$, form a group under addition.
Thus, we can consider the quotient $Q(R, G) = E(R, G)/\Hom(R, G)$.
The main goal can be rephrased as $Q(R, G) = 0$ for all $R$ and $G$.

To prove $Q(R, G) = 0$ by first principles, we would check that $f(x + y) = f(x) + f(y)$ for any $x, y \in E(R, G)$.
Now we fix $x$ and $y$, and consider the evaluation homomorphism $\phi : \Z \langle X, Y \rangle \to R$ given by $X \mapsto x$ and $Y \mapsto y$, where $\Z \langle X, Y \rangle$ is the ring of non-commutative integer polynomials with two variables.
Then one can check that $f \circ \phi \in E(\Z \langle X, Y \rangle, G)$, and if $f$ is a group homomorphism, then $f(x + y) = f(x) + f(y)$.
Thus, the goal reduces to proving $Q(\Z \langle X, Y \rangle, G) = 0$ for all groups $G$.
We can still make further reductions.

The first lemma says that $Q(R, G)$ is $6$-torsion for any $R$ and $G$.
Thus we can write $Q(R, G) = Q_2(R, G) \times Q_3(R, G)$, where $Q_p(R, G)$ is the set of $p$-torsion elements for $p = 2, 3$.
When $R$ is free as an abelian group, e.g. $R = \Z \langle X, Y \rangle$, we can reduce even further.
For $p = 2, 3$, by shifting with a group homomorphism, each element of $Q_p(R, G)$ has a representative that sends a fixed $\Z$-basis of $R$, say $\cB$, to zero.
Then $f$ is $p$-torsion as an element of $E(R, G)$, so in fact $f \in E(R, G[p])$, where $G[p]$ is the set of $p$-torsion elements of $G$.
If $f$ is non-zero, then we can choose an appropriate $\F_p$-linear functional $\alpha : G[p] \to \F_p$ such that $\alpha \circ f \in E(R, \F_p)$ is non-zero either.
Since it still sends the basis $\cB$ to zero, $\alpha \circ f$ is not a group homomorphism, which means $Q(R, \F_3) \neq 0$.
This means that the goal reduces entirely to two equalities: $Q_2(\Z \langle X, Y \rangle, \F_2) = 0$ and $Q_3(\Z \langle X, Y \rangle, \F_3) = 0$.

If $p = 3$, then the second lemma says more.
For any ring $R$ and $f \in E(R, \F_3)$ and $a, b \in R$, we have $f(3a + b) = 3 f(a) + f(b) = f(b)$.
In particular, $f$ naturally induces an excellent function in $E(R/(3), \F_3)$.
That is, we get $E(R, \F_3) \equiv E(R/(3), \F_3)$.
Thus, the equality $Q_3(\Z \langle X, Y \rangle, \F_3) = 0$ reduces further to $Q_3(\F_3[X, Y], \F_3) = 0$.





\subsection*{Basic results}

Here, we collect basic results regarding $\iota$-good functions, mainly those obtained by plug-and-chug.
An important observation is that if $f(a) = f(b)$ and $f(c) = f(d)$, then we have $f(a + c) = f (b + d) \iff f(ac) = f(bd)$.
Taking $c = d$ gives $f(a + c) = f(b + c) \iff f(ac) = f(bc) \iff f(ca) = f(cb)$ whenever $f(a) = f(b)$.
Taking $c = d = 1$, we get that $f(a) = f(b)$ implies $f(a + 1) = f(b + 1)$.
By induction on $n \in \N$, we get that $f(a) = f(b)$ implies $f(a + n) = f(b + n)$.
We also get that $f(a) = f(b)$ implies $f(na) = f(nb)$.

By taking $c = d = -1$ and replacing $a$ and $b$ with $a + 1$ and $b + 1$, we get that $f(a) = f(b)$ implies $f(-(a + 1)) = f(-(b + 1))$.
Then $f(a) = f(b)$ implies $f(-a) = f(-b)$, and in fact we get more:
\[ f(a) = f(b) \iff f(a + 1) = f(b + 1) \iff f(-a) = f(-b). \]
By induction, we get $f(a) = f(b) \iff f(a + n) = f(b + n)$ for any $n \in \Z$.

By plugging $x = y = 0$ into~\eqref{2017a6-eq0}, we get $f(\iota(f(0))^2) = 0$.
We now show:

\begin{theorem}
For any $c \in R$ such that $f(c) = 0$, we have $f(x + c) = f(x + 1)$ for any $x \in R$.
In particular, $f(1) = 0$, and if $f$ is non-periodic, then $f(c) = 0 \iff c = 1$ and so $\iota(f(0))^2 = 1$.
\end{theorem}
\begin{proof}
We first show that $f(c^2) = f(c^3) = 0$.
First, for any $d \in R$ such that $f(d) = 0$, plugging $y = d$ into~\eqref{2017a6-eq0} yields
\[ f(0) + f(x + d) = f(xd) \quad \forall x \in R. \]
Now we have $f(c^2) = f(0) + f(2c)$ and $f(c^3) = f(0) + f(c^2 + c) = 2 f(0) + f(2c + 1)$.
Thus our goal reduces to showing that $f(2c) = -f(0)$ and $f(2c + 1) = -2 f(0)$.
Since $f(c + 1) = -f(0)$, the former yields $f(2c) = f(c + 1)$, and so
\[ f(2c + 1) = f(c + 2) = f(2c) - f(0) = -f(0) - f(0) = -2 f(0). \]

We now show that $f(2c) = -f(0)$.
Since $f(c) = f(\iota(f(0))^2) = 0$, it suffices to show that $f(2 \iota(f(0))^2) = -f(0)$.
Indeed, we have $f(0) + f(c + 1) = f(c) = 0$, so $f(c + 1) = -f(0)$.
Plugging $x = 0$ and $y = c + 1$ into~\eqref{2017a6-eq0} yields $f(- \iota(f(0))^2) = 2 f(0)$.
Plugging $x = 0$ and $y = -\iota(f(0))^2$ into~\eqref{2017a6-eq0} yields $f(2 \iota(f(0))^2) = -f(0)$.

We now go back to the main goal.
We write down $f(xc^4)$ in two ways:
\[ f(xc^4) = f(0) + f(xc^2 + c^2) = f(0) + f((x + 1) c^2) = 2 f(0) + f(x + 1 + c^2), \]
\[ f(xc^4) = f(0) + f(xc + c^3) = f(0) + f((x + c^2) c) = 2 f(0) + f(x + c + c^2). \]
Replacing $x$ with $x - c^2$, we are done.
\end{proof}

Plugging $y = 1$ into~\eqref{2017a6-eq0} yields $f(0) + f(x + 1) = f(x)$ or $f(x + 1) = f(x) - f(0)$ for all $x \in R$.
Next, for any $x \in R$, we have $f(\iota(f(0)) \iota(f(x))) = f(0) - f(x)$.
Replacing $x$ with $\iota(f(0)) \iota(f(x))$, we get
\[ f(\iota(f(0))^2 - \iota(f(0)) \iota(f(x))) = f(\iota(f(0)) \iota(f(x) - f(0))) = f(x). \]
By the previous lemma,
\[ f(\iota(f(0))^2 - \iota(f(0)) \iota(f(x))) = f(1 - \iota(f(0)) \iota(f(x))) = f(-\iota(f(0)) \iota(f(x))) - f(0), \]
    and thus
\[ f(-\iota(f(0)) \iota(f(x))) = f(x) + f(0) = f(x - 1) \quad \forall x \in R. \tag{1.1}\label{2017a6-eq-basic1} \]
Since $f(a) = f(b)$ implies $f(-a) = f(-b)$, we get
\[ f(0) - f(x) = f(\iota(f(0)) \iota(f(x))) = f(1 - x). \tag{1.2}\label{2017a6-eq-basic2} \]
Then we get $f(x) + f(1 - x) = f(0)$, and thus
\[ f(x) + f(-x) = 2 f(0) \quad \forall x \in R. \tag{1.3}\label{2017a6-eq-basic3} \]

\begin{lemma}
Suppose that $f$ is non-periodic $\iota$-good.
Then for any $x \in R$, we have $\iota(f(0)) \iota(f(x)) = \iota(f(x)) = \iota(f(0))$.
\end{lemma}
\begin{proof}
Plug $(\iota(f(1 - x)) \iota(f(0)), \iota(f(0)) \iota(f(x)))$ in place of $(x, y)$ in~\eqref{2017a6-eq0}.
Using $\iota(f(0))^2 = 1$ and~\eqref{2017a6-eq-basic2}, we get
\[ f(\iota(f(x)) \iota(f(1 - x))) + f(\iota(f(1 - x)) \iota(f(0)) + \iota(f(0)) \iota(f(x))) = f(\iota(f(1 - x)) \iota(f(x))). \]
On the other hand,~\eqref{2017a6-eq0} with $x = 1 - y$ gives $f(\iota(f(1 - y)) \iota(f(y))) = f(y - y^2)$ for all $y \in R$.
Simplifying gives
\[ f(\iota(f(1 - x)) \iota(f(0)) + \iota(f(0)) \iota(f(x))) = 0 \iff \iota(f(1 - x)) \iota(f(0)) + \iota(f(0)) \iota(f(x)) = 1. \]
Since $f(x) + f(1 - x) = f(0)$, adding $\iota(f(x)) \iota(f(0))$ to the left hand side gives
\[ \iota(f(0))^2 + \iota(f(x)) \iota(f(0)) = \iota(f(0)) \iota(f(x)) + 1. \]
The lemma is proved, since $\iota(f(0))^2 = 1$.
\end{proof}

Finally, we classify injective non-periodic $\iota$-good functions.
This involves the theory of excellent functions.

\begin{lemma}\label{2017a6-nonperiodic-injective}
Suppose that all excellent functions $R \to G$ are group homomorphisms.\footnote{That is, $Q(R, G) = 0$, using the notation from the excellent function section.}
Suppose that $f$ is non-periodic $\iota$-good and injective.
Then there exists $a \in Z(R)$ with $a^2 = 1$ and a group homomorphism $\phi : R \to G$ with $\iota \circ \phi = \id_R$ such that $f(x) = \phi(a(1 - x))$ for all $x \in R$.
\end{lemma}
\begin{proof}
Take $a = \iota(f(0))$; we proved $a^2 = 1$.
By~\eqref{2017a6-eq-basic2} and injectivity, we get $a \iota(f(x)) = 1 - x$ and thus $\iota(f(x)) = a(1 - x)$ for all $x \in R$.
The previous lemma then yields $ax = xa$ for all $x \in R$, and thus $a \in Z(R)$.
Using the equation $\iota(f(x)) = a(1 - x)$,~\eqref{2017a6-eq0} simplifies to
\[ f((1 - x)(1 - y)) + f(x + y) = f(xy). \]
Then the function $x \mapsto f(1 - x)$ is excellent, and thus is a group homomorphism, say $\rho$.
Finally, choosing $\phi(x) = \rho(ax) = f(1 - ax)$ works.
\end{proof}





\subsection*{Period congruence: from general to non-periodic}

Here, we relate general good functions with non-periodic good functions.
The main result is as follows.

\begin{theorem}
The $\iota$-good functions are in an explicit bijection with the disjoint union of non-periodic $\phi \iota$-good functions across all double-sided ideals $I \subseteq R$, where $\phi : R \to R/I$ is the projection map.
\end{theorem}

More explicitly, the bijection is given by the following two results.
We will omit $\circ$ for this section at some equations for readability purposes.

\begin{theorem}
Let $I$ be a two-sided ideal of $R$ and $\phi : R \to R/I$ be the projection map.
Let $\iota : S \to R$ be a function, and let $f : R/I \to S$ be $\phi \iota$-good function.
Then $f \circ \phi : R \to S$ is an $\iota$-good function.
\end{theorem}
\begin{proof}
By~\eqref{2017a6-eq0} on $f$, for any $x, y \in R/I$,
\[ f(\phi \iota f(x) \; \phi \iota f(y)) + f(x + y) = f(xy). \]
Now replace $x$ and $y$ with $\phi(x)$ and $\phi(y)$ with $x, y \in R$:
\[ f(\phi \iota f \phi(x) \; \phi \iota f \phi(y)) + f(\phi(x) + \phi(y)) = f(\phi(x) \phi(y)). \]
Rearranging, we get that for any $x, y \in R$,
\[ f \phi(\iota f \phi(x) \; \iota f \phi(y)) + f \phi(x + y) = f \phi(xy). \]
This proves that $f \circ \phi$ is an $\iota$-good function.
\end{proof}

\begin{theorem}
Let $f : R \to S$ be an $\iota$-good function, and define
\[ I_f := \{c \in R : \forall x \in R, f(x + c) = f(x)\}. \]
Then $I_f$ is a double-sided ideal of $R$.

Furthermore, let $\phi : R \to R/I_f$ to be the usual projection map.
Then the reduction $\tilde{f} : R/I_f \to S$ of $f$ is a non-periodic $\phi \circ \iota$-good function.
\end{theorem}
\begin{proof}
Clearly, $I_f$ is a monoid under addition.
Since $R$ is a ring, it is group under addition.
So, it remains to show that $cy, yc \in I_f$ for any $c \in I_f$ and $y \in R$.

Comparing~\eqref{2017a6-eq0} using $y = c$ and $y = 0$ gives $f(xc) = f(0)$ for all $x \in R$.
Then replacing $y$ with $yc$ yields
\[ f(x) + f(\iota f(x) \iota f(0)) = f(0) = f(xyc) = f(x + yc) + f(\iota f(x) \iota f(yc)) = f(x + yc) + f(\iota f(x) \iota f(0)), \]
    so $f(x + yc) = f(x)$ for all $x, y \in R$.
That is, $yc \in I_f$ for any $y \in R$.
Similarly, we also get $cy \in I_f$ for any $y \in R$.
This shows that $I_f$ is a double-sided ideal.

Now write $f = \tilde{f} \circ \phi$.
Plugging into~\eqref{2017a6-eq0}, we get that for any $x, y \in R$,
\begin{align*}
    \tilde{f} \phi(\iota \tilde{f} \phi(x) \iota \tilde{f} \phi(y)) + \tilde{f} \phi(x + y) &= \tilde{f} \phi(xy) \\
    \tilde{f} (\phi \iota \tilde{f} \phi(x) \phi \iota \tilde{f} \phi(y)) + \tilde{f}(\phi(x) + \phi(y)) &= \tilde{f}(\phi(x) \phi(y)).
\end{align*}
Since $\phi$ is surjective, this implies that $\tilde{f}$ is $\phi \iota$-good.
Finally, one can check from the definition of $I_f$ that $\tilde{f}$ is non-periodic.
\end{proof}





\subsection*{Solution for subcases}

Here we prove lemmas that allow us to solve the problem for the desired subcases.
By Lemma~\ref{2017a6-nonperiodic-injective}, most of the work reduces to showing that $f$ is injective.

\begin{lemma}\label{2017a6-nonperiodic-2tf}
If $G$ is $2$-torsion free, then every non-periodic $\iota$-good function is injective.
\end{lemma}
\begin{proof}
Let $a, b \in R$ such that $f(a) = f(b)$.
Since $f(a) = f(b)$, $f(b) = f(a)$, and $f(a + b) = f(b + a)$, we get $f(ab) = f(ba)$.
Then we have $f(-ab) = f(-ba)$ and $f(-a) = f(-b)$.
This implies $f(a - b) = f(b - a)$.

Finally, substitute $x = a - b$ into~\eqref{2017a6-eq-basic3}.
We get $2 f(a - b) = 2 f(0)$.
Since $G$ is $2$-torsion free, we get
\[ f(a - b) = f(0) \iff f(a - b + 1) = 0 \iff a - b + 1 = 1 \iff a = b. \]
This proves that $f$ is injective.
\end{proof}

\begin{corollary}
Suppose that $G$ is $2$-torsion free and $3$-torsion free.
Then all non-periodic $\iota$-good functions are of form $x \mapsto \phi(a(1 - x))$ for some $a \in Z(R)$ with $a^2 = 1$ and group homomorphism $\phi : R \to G$ with $\iota \circ \phi = \id_R$.
All $\iota$-good functions are of form $x \mapsto x \mapsto \phi(a(1 - [x]))$ for some $a \in Z(R/I)$ with $a^2 = 1$ and group homomorphism $\phi : R/I \to G$ with $[\iota(\phi(x))] = x$ for all $x \in R/I$.
\end{corollary}

\begin{lemma}\label{2017a6-good-simple-ring-case}
If $R$ is simple, then $f$ is $\iota$-good iff either $f = 0$ or $f$ is non-periodic $\iota$-good.
\end{lemma}
\begin{proof}
The fact that $R$ is simple means that the period ideal $I_f$ of $f$ is either $R$ or $(0)$.
Thus $f$ is either constant, and so constant zero, or non-periodic, respectively.
\end{proof}

\begin{corollary}
If $R$ is simple, then the $\id_R$-good functions are $0$, $1 - x$, and $x - 1$.
\end{corollary}

\begin{lemma}\label{2017a6-integral-domain-char2-unit}
Suppose that $R$ is an integral domain of characteristic $2$.
Let $f$ be a non-periodic $\iota$-good function.
Then $\iota(f(x)) = 1 - x$ for any $x \in R^{\times}$.
\end{lemma}
\begin{proof}
From~\eqref{2017a6-eq0}, we get that for any units $a \in R^{\times}$,
\[ f(\iota(f(a + 1)) \iota(f(a^{-1} + 1))) = 0 \iff \iota(f(a + 1)) \iota(f(a^{-1} + 1)) = 1. \]
Thus, if $a$ is a unit, then $\iota(f(a + 1))$ is also a unit.
Since $R$ is an integral domain of characteristic $2$, we also have $\iota(f(0))^2 = 1 \iff \iota(f(0)) = 1$.
From this, it is easy to see that
\[ f(\iota(f(a + 1))) = f(\iota(f(0)) \iota(f(a + 1))) = f(0) - f(a + 1), \]
\[ (\iota \circ f)^2(a + 1) = \iota(f(0)) - \iota(f(a + 1)) = \iota(f(0)) + \iota(f(a + 1)) = \iota(f(a)). \]
One can check that $\iota(f(a)) = \iota(f(b))$ always implies $a = b$, namely, from comparing~\eqref{2017a6-eq0} using $y = 0$.
Thus we get $f(\iota(f(a + 1))) = f(a)$ for any unit $a$.
It remains to show the following: if $a, b \in R^{\times}$ satisfy $f(a) = f(b)$, then $a = b$.

Fix some $a, b \in R^{\times}$ with $f(a) = f(b)$.
Consider~\eqref{2017a6-eq0} with $(x, y) = (a + 1, b^{-1} + 1)$.
Since $f(1) = 0$, we get
\begin{align*}
    f((a + 1)(b^{-1} + 1)) &= f((a + 1) + (b^{-1} + 1)) \\
    f(ab^{-1} + a + b^{-1} + 1) &= f(a + b^{-1}) \\
    f(ab^{-1} + a + b^{-1}) &= f(1 + a + b^{-1}).
\end{align*}
By symmetry, we also have
\[ f(ba^{-1} + b + a^{-1}) = f(1 + b + a^{-1}). \]
Now notice the identity
\[ (1 + a + b^{-1})(1 + b + a^{-1}) = (ab^{-1} + a + b^{-1})(ba^{-1} + b + a^{-1}). \]
As a result, by plugging the appropriate values into~\eqref{2017a6-eq0}, we get
\[ f((1 + a + b^{-1}) + (1 + b + a^{-1})) = f((ab^{-1} + a + b^{-1}) + (ba^{-1} + b + a^{-1})). \]
Letting $C = a + b + 1$ and $D = a^{-1} + b^{-1} + 1$, the above equation is equivalent to saying that
\[ f(C + D) = f(CD + 1) = f(CD) + 1. \]
But then plugging $(x, y) = (C, D)$ into~\eqref{2017a6-eq0} yields
\[ f(f(C) f(D)) = 1 \iff f(C) f(D) = 0 \iff f(C) = 0 \vee f(D) = 0 \iff C = 1 \vee D = 1. \]
By definition of $C$ and $D$, and by $\rchar(R) = 2$, this is equivalent to
\[ a + b + 1 = 1 \vee a^{-1} + b^{-1} + 1 = 1 \iff a = b \vee a^{-1} = b^{-1}. \]
But $a^{-1} = b^{-1}$ yields $a = b$.
So, regardless, we have obtained $a = b$, as desired.
\end{proof}

\begin{corollary}
If $R$ is a field, then the $\id_R$-good functions are $0$ and $1 - x$.
\end{corollary}
