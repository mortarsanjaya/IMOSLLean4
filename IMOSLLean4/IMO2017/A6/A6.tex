Let $R$ be a ring, $S$ be an abelian (additive) group, and $\iota : S \to R$ be a group homomorphism.
Find all functions $f : R \to S$ such that for any $x, y \in R$,
\[ f(\iota(f(x)) \iota(f(y))) + f(x + y) = f(xy). \tag{*}\label{2017a6-eq0} \]

We say that $f : R \to S$ is \emph{$\iota$-good} if $f$ satisfies the above functional equation.
The main goal is actually to find only $\id_R$-good functions.
However, the given framework makes the problem slightly more tractable, even in the $\iota = \id_R$ case.

We say that $f$ is \emph{non-periodic $\iota$-good} if $f$ is $\iota$-good and $f$ has no non-zero period.
We say that $f$ is \emph{reduced $\iota$-good} if $f$ is non-periodic $\iota$-good and $\iota f(0) = 1$.
To classify $\iota$-good functions, we reduce to non-periodic and reduced cases.





\section{Basic results}

Here, we collect basic results regarding $\iota$-good functions, mainly those obtained by plug-and-chug.
By plugging $x = y = 0$ into~\eqref{2017a6-eq0}, we get $f(\iota f(0)^2) = 0$.
By plugging $y = 1$ into~\eqref{2017a6-eq0} and comparing, we get that $f(a) = f(b)$ implies $f(a + 1) = f(b + 1)$.
By induction, we get that for any $n \in \N$ and $a, b \in R$,
\[ f(a) = f(b) \implies f(a + n) = f(b + n). \]
Plugging $y = n$ into~\eqref{2017a6-eq0}, we also get that $f(a) = f(b)$ implies $f(na) = f(nb)$ for any $n \in \N$.
We now show:

\begin{theorem}
For any $c \in R$ such that $f(c) = 0$, we have $f(x + c) = f(x + 1)$ for any $x \in R$.
In particular, $f(1) = 0$, and if $f$ is non-periodic, then $f(c) = 0 \iff c = 1$.
\end{theorem}
\begin{proof}
We first show that $f(c^2) = f(c^3) = 0$.
First, for any $d \in R$ such that $f(d) = 0$, plugging $y = d$ into~\eqref{2017a6-eq0} yields
\[ f(0) + f(x + d) = f(xd) \quad \forall x \in R. \]
Now we have $f(c^2) = f(0) + f(2c)$ and $f(c^3) = f(0) + f(c^2 + c) = 2 f(0) + f(2c + 1)$.
Thus our goal reduces to showing that $f(2c) = -f(0)$ and $f(2c + 1) = -2 f(0)$.
Since $f(c + 1) = -f(0)$, the former yields $f(2c) = f(c + 1)$, and so
\[ f(2c + 1) = f(c + 2) = f(2c) - f(0) = -f(0) - f(0) = -2 f(0). \]

We now show that $f(2c) = -f(0)$.
Since $f(c) = f(\iota(f(0))^2) = 0$, it suffices to show that $f(2 \iota(f(0))^2) = -f(0)$.
Indeed, we have $f(0) + f(c + 1) = f(c) = 0$, so $f(c + 1) = -f(0)$.
Plugging $x = 0$ and $y = c + 1$ into~\eqref{2017a6-eq0} yields $f(- \iota(f(0))^2) = 2 f(0)$.
Plugging $x = 0$ and $y = -\iota(f(0))^2$ into~\eqref{2017a6-eq0} yields $f(2 \iota(f(0))^2) = -f(0)$.

We now go back to the main goal.
We write down $f(xc^4)$ in two ways:
\[ f(xc^4) = f(0) + f(xc^2 + c^2) = f(0) + f((x + 1) c^2) = 2 f(0) + f(x + 1 + c^2), \]
\[ f(xc^4) = f(0) + f(xc + c^3) = f(0) + f((x + c^2) c) = 2 f(0) +  f(x + c + c^2). \]
Replacing $x$ with $x - c^2$, we are done.
\end{proof}

Plugging $y = 1$ into~\eqref{2017a6-eq0} yields $f(0) + f(x + 1) = f(x)$ or $f(x + 1) = f(x) - f(0)$ for all $x \in R$.
Next, for any $x \in R$, we have $f(\iota(f(0)) \iota(f(x))) = f(0) - f(x)$.
Replacing $x$ with $\iota(f(0)) \iota(f(x))$, we get
\[ f(\iota(f(0))^2 - \iota(f(0)) \iota(f(x))) = f(\iota(f(0)) \iota(f(x) - f(0))) = f(x). \]
By the previous lemma,
\[ f(\iota(f(0))^2 - \iota(f(0)) \iota(f(x))) = f(1 - \iota(f(0)) \iota(f(x))) = f(-\iota(f(0)) \iota(f(x))) - f(0), \]
    and thus \[ f(-\iota(f(0)) \iota(f(x))) = f(x) + f(0) = f(x - 1) \quad \forall x \in R. \tag{1.1}\label{2017a6-eq1-1} \]
Next, by plugging $y = -1$ into~\eqref{2017a6-eq0}, we get
\[ f(\iota(f(x)) \iota(f(-1))) + f(x) + f(0) = f(-x) \quad \forall x \in R. \]
More importantly, if $f(a) = f(b)$, then the above yields $f(-a) = f(-b)$.
As a result, we get $f(0) - f(x) = f(\iota(f(0)) \iota(f(x))) = f(1 - x)$, or
\[ f(x) + f(1 - x) = f(0) \quad \forall x \in R. \tag{1.2}\label{2017a6-eq1-2} \]





\section{Period congruence: from general to non-periodic}

Here, we relate general good functions with non-periodic good functions.
The main result is as follows.

\begin{theorem}
The $\iota$-good functions are in an explicit bijection with the disjoint union of non-periodic $\phi \iota$-good functions across all double-sided ideals $I \subseteq R$, where $\phi : R \to R/I$ is the projection map.
\end{theorem}

More explicitly, the bijection is given by the following two results.
We will omit $\circ$ for this section at some equations for readability purposes.

\begin{theorem}
Let $I$ be a two-sided ideal of $R$ and $\phi : R \to R/I$ be the projection map.
Let $\iota : R \to S$ be a function, and let $f : R/I \to S$ be $\phi \iota$-good function.
Then $f \circ \phi : R \to S$ is an $\iota$-good function.
\end{theorem}
\begin{proof}
By~\eqref{2017a6-eq0} on $f$, for any $x, y \in R/I$,
\[ f(\phi \iota f(x) \; \phi \iota f(y)) + f(x + y) = f(xy). \]
Now replace $x$ and $y$ with $\phi(x)$ and $\phi(y)$ with $x, y \in R$:
\[ f(\phi \iota f \phi(x) \; \phi \iota f \phi(y)) + f(\phi(x) + \phi(y)) = f(\phi(x) \phi(y)). \]
Rearranging, we get that for any $x, y \in R$,
\[ f \phi(\iota f \phi(x) \; \iota f \phi(y)) + f \phi(x + y) = f \phi(xy). \]
This proves that $f \circ \phi$ is an $\iota$-good function.
\end{proof}

\begin{theorem}
Let $f : R \to S$ be an $\iota$-good function, and define
\[ I_f := \{c \in R : \forall x \in R, f(x + c) = f(x)\}. \]
Then $I_f$ is a double-sided ideal of $R$.

Furthermore, let $\phi : R \to R/I_f$ to be the usual projection map.
Then the reduction $\tilde{f} : R/I \to S$ of $f$ is a non-periodic $\phi \circ \iota$-good function.
\end{theorem}
\begin{proof}
Clearly, $I_f$ is a monoid under addition.
Since $R$ is a ring, it is group under addition.
So, it remains to show that $cy, yc \in I_f$ for any $c \in I_f$ and $y \in R$.

Comparing~\eqref{2017a6-eq0} using $y = c$ and $y = 0$ gives $f(xc) = f(0)$ for all $x \in R$.
Then replacing $y$ with $yc$ yields
\[ f(x) + f(\iota f(x) \iota f(0)) = f(0) = f(xyc) = f(x + yc) + f(\iota f(x) \iota f(yc)) = f(x + yc) + f(\iota f(x) \iota f(0)), \]
    so $f(x + yc) = f(x)$ for all $x, y \in R$.
That is, $yc \in I_f$ for any $y \in R$.
Similarly, we also get $cy \in I_f$ for any $y \in R$.
This shows that $I_f$ is a double-sided ideal.

Now write $f = \tilde{f} \circ \phi$.
Plugging into~\eqref{2017a6-eq0}, we get that for any $x, y \in R$,
\begin{align*}
    \tilde{f} \phi(\iota \tilde{f} \phi(x) \iota \tilde{f} \phi(y)) + \tilde{f} \phi(x + y) &= \tilde{f} \phi(xy) \\
    \tilde{f} (\phi \iota \tilde{f} \phi(x) \phi \iota \tilde{f} \phi(y)) + \tilde{f}(\phi(x) + \phi(y)) &= \tilde{f}(\phi(x) \phi(y)).
\end{align*}
Since $\phi$ is surjective, this implies that $\tilde{f}$ is $\phi \iota$-good.
Finally, one can check from the definition of $I_f$ that $\tilde{f}$ is non-periodic.
\end{proof}







...
