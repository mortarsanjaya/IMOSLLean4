Define the function $f : \N \to \N$ by
\[ f(n) = \begin{cases} \sqrt{n}, & \text{if $n$ is a square,} \\ n + 3, & \text{otherwise.} \end{cases} \]
Determine all values of $N$ such that there exists $a \in \N$ such that $f^n(N) = a$ for infinitely many values of $n$.



\subsection*{Answer}

$1$ and multiples of $3$.



\subsection*{Solution}

Official solution: \url{http://www.imo-official.org/problems/IMO2017SL.pdf}

We generalize the official solution by building a more general theory.
We say that $N \in \N$ is \emph{good} if it satisfies the problem's condition.
Note that $N$ is good iff the sequence $(f^k(N))_{k \geq 0}$ is eventually periodic.
For convenience, denote $p = 3$.
The main lemmas do not assume the value of $p$.

\begin{lemma}\label{2017n1-1}
For any $N \in \N$, $f(N)$ is good iff $N$ is good.
In particular, for any $k \in \N$, $f^k(N)$ is good iff $N$ is good.
\end{lemma}
\begin{proof}
Immediate from the definition of good numbers.
\end{proof}

\begin{lemma}\label{2017n1-2}
If $p > 0$ and $N$ is good, then $N$ is a square mod $p$.
\end{lemma}
\begin{proof}
Suppose for the sake of contradiction that $N$ is not a square mod $p$.
Then for any $k \in \N$, $N + kp$ is not a square mod $p$, and thus not a square.
By induction, we get $f^k(N) = N + kp$ for all $k \in \N$.
The sequence $(f^k(N))_{k \geq 0}$ is not eventually periodic; a contradiction.
\end{proof}

\begin{lemma}\label{2017n1-3}
For any $N, k \in \N$, either $f^k(N) = N + pk$ or $f^m(N) \leq \sqrt{N + pk}$ for some $m \leq n$.
In particular, if $x^2 \geq N$ and $x^2 \equiv N \pmod{p}$ for some $x$, then $f^m(N) \leq x$ for some $m$.
\end{lemma}
\begin{proof}
For the first statement, induction on $k$; the base case $k = 0$ is clear.
Now fix $k$ and suppose that either $f^k(N) = N + pk$ or $f^m(N) \leq \sqrt{N + pk}$ for some $m \leq n$.
In the former case, $f^{k + 1}(N)$ is either equal to $N + p(k + 1)$ or $\sqrt{N + pk} \leq \sqrt{N + p(k + 1)}$.
In the latter case, $f^m(N) \leq \sqrt{N + p(k + 1)}$.
Induction step is complete.

For the second statement, we can write $x^2 = N + pk$ for some $x$.
Then we are done by the first claim with $k = m$.
(If $f^k(N) = N + pk = x^2$, take $m = k + 1$.)
\end{proof}

\begin{lemma}\label{2017n1-4}
Let $p > 0$ and $N$ be a non-negative integer such that $N$ is a square mod $p$.
\begin{itemize}
    \item   If $N > 2p$, then there exists $k \in \N$ such that $f^k(N) < N$.
    \item   If $N \leq 2p$ and $p > 2$, then then there exists $k \in \N$ such that $f^k(N) \leq p$.
\end{itemize}
\end{lemma}
\begin{proof}
For the first part, there exists $x \in \N$ such that $\sqrt{N} \leq x \leq \sqrt{N} + p$ and $x^2 \equiv N \pmod{p}$.
By Lemma~\ref{2017n1-3}, $f^k(N) \leq x \leq \sqrt{N} + p$ for some $k$.
Since $N > 2p$, we get $\sqrt{N} + p < N$.

For the second part, we need to find $x \leq p$ such that $x^2 \geq N$ and $x^2 \equiv N \pmod{p}$.
First pick any $x \leq p$ such that $x^2 \equiv N \pmod{p}$.
If $N < p$, the mod condition clearly implies $x^2 \geq N$ and we are done.
If $N = 2p$, then take $x = p$.
So now, suppose that $p \leq N < 2p$ and $x^2 < N$.
Since $x^2 < N$ and $x^2 \equiv N \pmod{p}$, we have $x^2 < p$.
One can show that $p - x > x$, and thus $(p - x)^2 \geq x^2 + p$ since $(p - x)^2 \equiv x^2 \equiv N \pmod{p}$.
Since $N < 2p$, this already yields $(p - x)^2 \geq N$, as desired.
\end{proof}

\begin{lemma}\label{2017n1-5}
If $p > 2$ and $N$ is good, then $f^k(N) \leq p$ for some $k$.
\end{lemma}
\begin{proof}
Immediate by induction on $N$ due to Lemma~\ref{2017n1-4}.
\end{proof}

\begin{lemma}\label{2017n1-6}
If $N$ is good and $p \nmid N$, then $p \nmid f(N)$.
\end{lemma}
\begin{proof}
We have either $f(N) = N + p$ or $f(N)^2 = N$.
The lemma is immediate in both cases.
\end{proof}

\begin{lemma}\label{2017n1-7}
We have $f(N) = 0 \iff N = 0$ and $f(N) = 1 \iff N = 1$.
In particular, for any $k \in \N$, we have $f^k(N) = 0 \iff N = 0$ and $f^k(N) = 1 \iff N = 1$.
\end{lemma}
\begin{proof}
Both results are done by bashing using the fact that either $f(N) = N + p$ or $f(N)^2 = N$.
\end{proof}

Since $2$ is not a square mod $3$, $2$ is not good for $p = 3$ by Lemma~ref{2017n1-2}.
If $N$ is good, Lemma~\ref{2017n1-5} says that $f^k(N) \leq 3$ for some $k$.
If $N > 1$, then Lemma~\ref{2017n1-7} implies $f^k(N) > 1$.
Since $2$ is not good, this forces $f^k(N) = 3$.
By Lemma~\ref{2017n1-6}, this forces $3 \mid N$.
It remains to show the converse.

\begin{lemma}\label{2017n1-8}
If $p$ is squarefree and $p \mid N$, then $p \mid f(N)$.
In particular, if $p \mid N$, then $N$ is good.
\end{lemma}
\begin{proof}
We have either $p \mid f(N) = N + p$ or $p \mid N = f(N)^2$.
Since $p$ is squarefree, the latter still yields $p \mid f(N)$.
Since $p \mid N$, this yields $N \in \{0, p, 2p\}$.
Since $0$ is good, we just have to check that $p$ and $2p$ are good.
The case $p = 1$ can be checked by hand, so now assume that $p > 1$.

Since $p > 1$ is squarefree, $p$ is not a square, we have $f(p) = 2p$ and it remains to show that $p$ is good.
By induction, $f^k(p) = p(k + 1)$ for all $k < p$.
Thus, $f^{p - 1}(p) = p^2$ and $f^p(p) = p$.
\end{proof}



\subsection*{Extra notes}

It is also possible to prove a strong converse to Lemma~\ref{2017n1-8}.
If $p$ is not squarefree, and $N$ is a positive multiple of $p$, then $N$ is not good.

\begin{lemma}
Let $q$ be a prime factor of $p$ and $N \in \N$ such that $\nu_q(N) < \nu_q(p)$.
Then $N$ is not good.
\end{lemma}
\begin{proof}
If $N$ is good, then $N$ is a square mod $p$.
However, for any $x \in \N$ such that $x^2 \equiv N \pmod{p}$, we have $\nu_q(x) = \nu_q(N)/2 < \nu_q(N)$.
So, there exists $k$ such that $\nu_q(f^k(N)) < \nu_q(N)$.
Repeat the same argument with $f^k(N)$ and so on, and we get a strictly decreasing sequence of non-negative integers.
A contradiction!
\end{proof}

\begin{lemma}
Suppose that $p$ is NOT squarefree.
Let $N > 0$ be an integer such that $p \mid N$.
Then $N$ is not good.
\end{lemma}
\begin{proof}
Suppose for the sake of contradiction that $N$ is good.
By Lemma~\ref{2017n1-5}, there exists $k$ such that $f^k(N) \leq p$.
Since $p$ is not squarefree, it can be written as $p_1^2 p_2$ with $p_1 > 1$ and $p_2$ squarefree.
Then one can show that $f^k(p) = p(k + 1)$ for all $k < p_2$ and then $f^{p_2}(p) = p_1 p_2 < p$.
Thus, there exists $k$ such that $f^k(N) < p$.

Lemma~\ref{2017n1-6} can be generalized to show that if $q \mid p$ and $q \mid f(N)$, then $q \mid N$.
Thus, any prime divisor of $p$ also divides $f^k(N)$.
Since $f^k(N) < p$, there exists a prime number $q$ such that $\nu_q(f^k(N)) < \nu_q(p)$.
By the previous lemma, $f^k(N)$ is not good; a contradiction.
\end{proof}

In general, it is not true that the only good numbers are $1$ and multiples of $p$, even when $p$ is prime.
For example, $2$ is good for $p = 127$.
