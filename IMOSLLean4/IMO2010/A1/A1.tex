\documentclass{article}

\usepackage{fullpage}
\usepackage{amsmath, amsfonts, amssymb, amsthm}
\usepackage{hyperref}

\setlength{\parindent}{0pt}
\setlength{\parskip}{5pt}

\newcommand{\Z}{\mathbb{Z}}

\title{IMO 2010 A1 (P1)}
\author{}
\date{}

\begin{document}

\maketitle



\subsection*{Problem}

A \emph{floor function} $\lfloor \cdot \rfloor : R \to \Z$ on a totally ordered ring $R$ is a function such that, for any $r \in R$ and $n \in \Z$, $n \leq \lfloor r \rfloor$ if and only if $n \leq r$ in $R$.

Let $F$ and $R$ be a totally ordered field and a totally ordered ring, respectively, both equipped with a floor function.
Find all functions $f : F \to R$ such that
\[ f(\lfloor x \rfloor y) = f(x) \lfloor f(y) \rfloor \quad \text{for all } x, y \in F. \tag{*}\label{2010a1-eq0} \]



\subsection*{Solution}

Official solution: \url{https://www.imo-official.org/problems/IMO2010SL.pdf}

Although we start with the same substitution as Solution 1, we make a shortcut in the case $f(0) = 0$.

First, plugging $x = y = 0$ into~\eqref{2010a1-eq0} yields $f(0) = f(0) \lfloor f(0) \rfloor$.
Thus either $\lfloor f(0) \rfloor = 1$ or $f(0) = 0$.
We now split into two cases.

In the former case, plugging $y = 0$ into~\eqref{2010a1-eq0} yields that $f$ is constant.
It is easy to check that $f \equiv C$ works for any $C \in R$ with $1 \leq C < 2$.

In the latter case, we claim that $f \equiv 0$.
Plugging $x = 1$ into~\eqref{2010a1-eq0} yields
\[ f(y) = f(1) \lfloor f(y) \rfloor \quad \text{for all } y \in F. \]
Thus, to prove $f \equiv 0$, it suffices to show that $f(1) = 0$.

Plugging $x = y = 1/2$ into~\eqref{2010a1-eq0} yields
\[ f(1/2) \lfloor f(1/2) \rfloor = f(0) = 0 \implies \lfloor f(1/2) \rfloor = 0. \]
Then plugging $(x, y) = (2, 1/2)$ into~\eqref{2010a1-eq0} yields $f(1) = 0$, as desired.



\end{document}
