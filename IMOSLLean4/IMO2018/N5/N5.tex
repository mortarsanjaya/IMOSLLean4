Determine whether there exists $x, y, z, t \in \N^+$ such that $xy - zt = x + y = z + t$ and both $xy$ and $zt$ are perfect squares.



\subsection*{Answer}

No.



\subsection*{Solution}

Official solution: \url{https://www.imo-official.org/problems/IMO2018SL.pdf}

We use Solution 2 of the official solution.
In fact, we do more; we characterize all $(x, y, z, t) \in \Z^4$ satisfying the given condition.
We say that $(x, y, z, t) \in \Z^4$ is \emph{good} if $xy - zt = x + y = z + t$.

\begin{claim}
A quadruple $(x, y, z, t) \in \Z^4$ is good if and only if there exists $k, \ell \in \Z$ such that
\[ (x, y, z, t) = (2k \ell - k + \ell, 2k \ell + k - \ell, 2k \ell + k + \ell, 2k \ell - k - \ell). \]
\end{claim}
\begin{proof}
By tedious calculation, one can show that the given quadruples are all good.
For the converse, first notice that $xy - zt = x + y = z + t$ cannot be odd.
Indeed, if $x + y = z + t$ is odd, then $xy$ and $zt$ are even, so $xy - zt$ is even.

Now write $x + y = z + t = xy - zt = 2s$ for some integer $s$.
Both $x - y$ and $z - t$ are also even, so we can write $x - y = 2p$ and $z - t = 2q$ for some integers $p$ and $q$.
Then $xy = s^2 - p^2$ and $zt = s^2 - q^2$, so $2s = q^2 - p^2$.
This means $q + p$ and $q - p$ are both even, so we can write
\[ q + p = 2k, \quad q - p = 2 \ell, \quad s = 2k \ell \]
    for some $k, \ell \in \Z$.
Plugging back the equations for $x, y, z, t$ proves the claim.
\end{proof}

\begin{claim}
Given a good quadruple $(x, y, z, t) \in \Z^4$, $xyzt$ is a square iff either one of the following holds:
\begin{itemize}
    \item   $y = -x$ and $t = -z$; or
    \item   $(x, y, z, t)$ is equal to $(2, 2, 4, 0)$, $(2, 2, 0, 4)$, $(-4, 0, -2, -2)$, or $(0, -4, -2, -2)$.
\end{itemize}
\end{claim}
\begin{proof}
By the previous claim, there exists $k, \ell \in \Z$ such that
\[ (x, y, z, t) = (2k \ell - k + \ell, 2k \ell + k - \ell, 2k \ell + k + \ell, 2k \ell - k - \ell). \]
Then $xyzt = ((2k \ell)^2 - (k^2 + \ell^2))^2 - (2k \ell)^2$.
Since $(2k \ell)^2 \geq k^2 + \ell^2 \geq 0$, this forces either $2 k \ell = 0$ or
\[ (2k \ell)^2 \geq 2((2k \ell)^2 - (k^2 + \ell^2)) - 1 \iff 2 k^2 \ell^2 + 1/2 \geq 4 k^2 \ell^2 - (k^2 + \ell^2) \iff 2 k^2 \ell^2 \leq k^2 + \ell^2 + 1. \]
The latter forces $(k^2, \ell^2) = (1, 1)$.
Bashing all four cases gives the four possible quadruples on the second option.
\end{proof}

Now we go back to the main problem.
In the first case, one of $x$ or $y$ is non-positive.
In the second case, one of $x$, $y$, $z$, or $t$ equals zero.
In both cases, the four integers cannot be all positive.



\subsection*{Extra notes}

In fact, the claim shows that if $xy$ and $zt$ are both squares, then $(x, y, z, t)$ is either $(0, 0, 0, 0)$ or one of the four special quadruplets.
In all cases, we also have $xyzt = 0$.