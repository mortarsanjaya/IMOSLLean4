Let $R$ be a ring with floor, and let $f : R \to R$ be the function defined by
\[ f(x) = \lfloor x \rfloor (x - \lfloor x \rfloor). \]
If $R$ is archimedean, prove that for any $r \in R$, there exists $N \in \N$ such that $f^{k + 2}(r) = f^k(r)$ for all $k \geq N$.





\subsection*{Solution}

Official solution: \url{https://www.imo-official.org/problems/IMO2006SL.pdf}

We follow the official solution.
We show even more; either one of the following holds:
\begin{itemize}
    \item   The sequence $(f^k(r))_{k \geq 0}$ eventually becomes zero;
    \item   The sequence $(f^k(r))_{k \geq 0}$ eventually becomes $-r, r - 1, -r, r - 1, \ldots$ for some $0 < r < 1$; or
    \item   There exists a positive integer $m > 1$ and $N$, and an infinitesimal element $\varepsilon$ such that for all $k \geq 0$,
            \[ (C + 1) f^{N + k}(r) = -m^2 + (-m)^k \varepsilon. \]
            By infinitesimal, we mean $k |\varepsilon| < 1$ for all $k \in \N$.
\end{itemize}

On archimedean floor rings, since they have no non-zero infinitesimal elements, the third case implies that $(f^k(r))_{k \geq 0}$ is eventually constant.
Combined with the remaining two cases, we would get the desired result.

We start with some observations.
If $r \geq 0$, then $0 \leq f(r) \leq \lfloor r \rfloor$.
In particular, if $r \geq 0$, then the sequence $(\lfloor f^k(r) \rfloor)_{k \geq 0}$ is a non-increasing sequence of non-negative integers.
Thus, this sequence must be eventually constant.
On the other hand, if $r \geq 1$, then we actually have $f(r) < \lfloor r \rfloor$.
So, the sequence must eventually becomes zero.
If $\lfloor f^n(r) \rfloor = 0$, then $f^{n + 1}(r) = 0$, so $(f^k(r))_{k \geq 0}$ eventually becomes zero.

Now suppose that $r < 0$.
In this case, we have $\lfloor r \rfloor \leq f(r) \leq 0$.
So the sequence $(\lfloor f^k(r) \rfloor)_{k \geq 0}$ is a non-decreasing sequence of non-positive integers.
Similar to the previous case, this sequence is eventually constant, say $-C$ for some $C \geq 0$.
If $C = 0$, then again $(f^k(r))_{k \geq 0}$ is eventually zero.

If $C = 1$, note that for any $s \in R$ such that $\lfloor s \rfloor = -1$, we have $f(s) = 1 - s$.
Now choose any $N \in \N$ such that $\lfloor f^n(r) \rfloor = -1$ for all $n \geq N$.
Then we get $f^{n + 1}(r) = 1 - f^n(r)$ for all $n \geq N$; this is a desired form.

Finally, suppose that $C > 1$.
Choose any $N \in \N$ such that $\lfloor f^n(r) \rfloor = -C$ for all $n \geq N$.
For any $s \in R$ such that $\lfloor s \rfloor = -C$, one can check that
\[ f(s) = -C(C - s) \implies (C + 1) f(s) + C^2 = -C ((C + 1) s + C^2). \]
Thus, by induction on $k$, one can see that for any $k \geq 0$,
\[ (C + 1) f^{k + N}(r) + C^2 = (-C)^k ((C + 1) f^N(r) + C^2). \]
Since $C > 1$, this is possible only if $\varepsilon = (C + 1) f^N(r) + C^2$ is an infinitesimal.
We get $(C + 1) f^{k + N}(r) = -C^2 + (-C)^k \varepsilon$ for all $k \in \N$.
If $R$ is archimedean, then $\varepsilon = 0$ and we get $(C + 1) f^k(r) = -C^2$ for all $k \geq N$.
In particular, $C$ is invertible in $R$ and the sequence $(f^k(r))_{k \geq 0}$ is eventually constant.





\subsection*{Implementation details}

One part of the solution can be done with a shortcut.
Instead of proving $f(r) \leq \lfloor r \rfloor$ for $r \geq 0$ and $\lfloor r \rfloor \leq f(r)$ for $r < 0$, we can just prove $|\lfloor f(r) \rfloor| \leq |\lfloor r \rfloor|$ for all $r \in R$.
Then the sequence $(|\lfloor f^k(r) \rfloor|)_{k \geq 0}$ is non-increasing and hence eventually constant.
The rest of the solution follow accordingly.
