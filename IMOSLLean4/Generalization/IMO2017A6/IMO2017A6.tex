Let $R$ be a ring.
Find all functions $f : R \to R$ such that for any $x, y \in R$,
\[ f(f(x) f(y)) + f(x + y) = f(xy). \tag{*}\label{2017a6-eq0} \]



\subsection*{Main Progress}

Let $I$ be a double-sided ideal of $R$ and $[\cdot] : R \to R/I$ be the projection map.
Let $\phi : R/I \to R$ be a group homomorphism such that $[\phi(x)] = x$ for all $x \in R/I$.
Let $a$ be an element of the centre $Z(R/I)$ of $R/I$ such that $a^2 = 1$.
Then the function $f(x) = \phi(a(1 - [x]))$ satisfies~\eqref{2017a6-eq0}.
If $2$ and $3$ are non-zero-divisors of $R$, then there are no other functions satisfying~\eqref{2017a6-eq0}.

\textbf{Note.}
There is a case where $I \neq 0$.
Let $S$ be an arbitrary ring.
Set $R = S[X]$ and define $f : R[X] \to R[X]$ by $f(P) = (1 - P(1)) X$ for all $P \in R[X]$.
One can check that $I = \langle X - 1 \rangle$, $\iota(P) = XP$, and $a = 1$ works.



\subsection*{Solution}

The solution in \href{https://artofproblemsolving.com/community/c6h1480146p8693244}{this AoPS thread}, post \#75, given by \textbf{anantmudgal09}, works when $R$ is a field of characteristic $2$.
Under a very small change, it also works when commutativity assumption is removed from $R$.
We extend this solution to work over more general rings.
See the third claim.

It is easy to check that the functions given in the previous subsection are all satisfies~\eqref{2017a6-eq0}.
The main problem is proving the converse.
In doing so, we need to generalize the functional equation as follows.

\begin{problem}
Let $R$ be a ring and let $\iota : G \to R$ be a group homomorphism, where $G$ is an (additive) abelian group.
Find all functions $f : R \to G$ such that for any $x, y \in R$,
\[ f(\iota(f(x)) \iota(f(y))) + f(x + y) = f(xy). \tag{**}\label{2017a6-eq00} \]
\end{problem}
\begin{proof}[Answer]
Let $I$ be a double-sided ideal of $R$ and $[\cdot] : R \to R/I$ be the projection map.
Let $\phi : R/I \to G$ be a group homomorphism such that $[\iota(\phi(x))] = x$ for all $x \in R/I$.
Let $a \in Z(R/I)$ such that $a^2 = 1$.
Then the function $f(x) = \phi(a(1 - [x]))$ satisfies~\eqref{2017a6-eq00}.
If $G$ is $2$- and $3$-torsion free, then there are no other functions satisfying~\eqref{2017a6-eq00}.
\end{proof}

This recovers the answer for~\eqref{2017a6-eq0} by setting $\iota$ to be the identity map.
We now solve~\eqref{2017a6-eq00}.

We say that a function $f : R \to G$ is $\iota$-\textbf{good} if it satisfies~\eqref{2017a6-eq00}.
We say, furthermore, that $f$ is \textbf{non-periodic} if any elements $c, d \in R$ satisfying $f(x + c) = f(x + d)$ for all $x \in R$ also satisfies $c = d$.
We start with a reduction from $\iota$-good functions to non-periodic $\iota$-good functions.
Before that, we note the following property: for any $a, b, c, d \in R$ such that $f(a) = f(b)$ and $f(c) = f(d)$, we have $f(a + c) = f(b + d)$ if and only if $f(ac) = f(bd)$.

\begin{claim}
Let $f$ be an $\iota$-good function, and denote
\[ I_f = \{c \in R : \forall x \in R, f(x + c) = f(x)\}. \]
Then $I_f$ is a double-sided ideal of $R$.
Furthermore, let $\tilde{f} : R/I \to G$ be the quotient map induced by $f$.
Then $\tilde{f}$ is $[\cdot] \circ \iota$-good and non-periodic, where $[\cdot] : R \to R/I_f$ is the projection map.
\end{claim}
\begin{proof}
The only part of the claim that is not immediate is that $I_f$ is closed under multiplication from both sides.
To prove this, take any $c \in I_f$ and $t \in R$.
Comparing~\eqref{2017a6-eq00} with $(x, y) = (c, y)$ and $(x, y) = (0, y)$ gives $f(cy) = f(0)$ for all $y \in R$.
Comparing~\eqref{2017a6-eq00} with $(x, y) = (ct, y)$ and $(x, y) = (0, y)$ then yields
\[ f(\iota(f(0)) \iota(f(y))) + f(y) = f(0) = f(cty) = f(\iota(f(0)) \iota(f(y))) + f(ct + y). \]
As this holds for any $y \in R$, we get $ct \in I_f$, as desired.
By symmetry, we also get $tc \in I_f$.
\end{proof}

As a result, we may make the assumption that $f$ is non-periodic whenever necessary.
We will not make this assumption for now.
We continue with more observations.

Consider $a, b \in R$ arbitrary with $f(a) = f(b)$.
Then comparing~\eqref{2017a6-eq00} with $(x, y) = (a, 1)$ and $(x, y) = (b, 1)$ gives $f(a + 1) = f(b + 1)$.
Applying the argument once more gives $f(a + 2) = f(b + 2)$, and plugging into~\eqref{2017a6-eq00} gives $f(2a) = f(2b)$.
Meanwhile, comparing~\eqref{2017a6-eq00} with $(x, y) = (a + 1, -1)$ and $(x, y) = (b + 1, -1)$ gives $f(-a) = f(-b)$.
In particular, we get $f(-a) = f(-b)$ if and only if $f(a) = f(b)$.

We now show that $f(1) = 0$ and more: if $f(c) = 0$, then $f(x + c) = f(x + 1)$ for any $x \in R$.
First, note that plugging $x = y = 0$ into~\eqref{2017a6-eq00} yields $f(\iota(f(0))^2) = 0$.

\begin{claim}
If $f(c) = 0$, then $f(x + c) = f(x + 1)$ for any $x \in R$.
\end{claim}
\begin{proof}
For convenience, denote $C = \iota(f(0))$.
For any $d \in R$ such that $f(d) = 0$, plugging $y = d$ into~\eqref{2017a6-eq00} yields
\[ f(0) + f(x + d) = f(xd). \tag{1}\label{2017a6-eq1} \]
In particular, we also get $f(d + 1) = -f(0)$.
Since $f(C^2) = 0$, plugging $(x, y) = (0, C^2 + 1)$ into~\eqref{2017a6-eq00} yields $f(-C^2) = 2f(0)$.
Now plugging $(x, y) = (0, -C^2)$ into~\eqref{2017a6-eq00} yields $f(2C^2) = -f(0)$.
From the latter, we also get that $f(d) = 0$ implies $f(2d) = -f(0)$.

Now fix some $c \in R$ such that $f(c) = 0$.
By the previous paragraph, we have $f(2c) = -f(0)$.
Then~\eqref{2017a6-eq1} yields $f(c^2) = f(0) + f(2c) = 0$.
Since $f(c) = 0$,~\eqref{2017a6-eq1} yields $f(c + 1) = -f(0) = f(2c)$, and so
\[ f(2c + 1) = f(c + 2) = f(2c) - f(0) = -2 f(0). \]
Thus we get $f(c^3) = f(0) + f(c^2 + c) = 2f(0) + f(2c + 1)$.
In summary, we get $f(c^2) = f(c^3) = 0$.

Finally, we write $f(xc^4)$ in two ways.
By~\eqref{2017a6-eq1} and the fact that $f(c^2) = f(c^3) = 0$, we get
\[ f(xc^4) = f(0) + f(xc + c^3) = f(0) + f((x + c^2) c) = 2 f(0) + f(x + c + c^2), \]
\[ f(xc^4) = f(0) + f(xc^2 + c^2) = f(0) + f((x + 1) c^2) = 2 f(0) + f(x + 1 + c^2). \]
Replacing $x$ with $x - c^2$, we get $f(x + c) = f(x + 1)$ for all $x \in R$, as desired.
\end{proof}

In particular, we get $f(1) = 0$ as well.
Plugging $y = 1$ into~\eqref{2017a6-eq00} now yields $f(x + 1) = f(x) - f(0)$.
Replacing $x$ with $x - 1$, we get
\[ f(x - 1) = f(x) + f(0). \tag{2}\label{2017a6-eq2} \]
For any $t \in R$, by plugging $(x, y) = (0, \iota(f(0)) \iota(f(t)))$ into~\eqref{2017a6-eq00} and then applying~\eqref{2017a6-eq00} gives
\[ f(1 - \iota(f(0)) \iota(f(t))) = f(t). \tag{3}\label{2017a6-eq3} \]
Then applying~\eqref{2017a6-eq2} gives us
\[ f(\iota(f(0)) \iota(f(t))) + f(0) = f(\iota(f(0)) \iota(f(t)) - 1) = f(-t). \]
Applying~\eqref{2017a6-eq00} again then gives us
\[ f(t) + f(-t) = 2 f(0). \tag{4}\label{2017a6-eq4} \]

\begin{claim}
Suppose that $G$ is $2$-torsion free.
Then any non-periodic $\iota$-good function $f : R \to G$ is injective.
\end{claim}
\begin{proof}
Fix $a, b \in R$ such that $f(a) = f(b)$.
By~\eqref{2017a6-eq00}, we get $f(ab) = f(ba)$.
Then we also get $f(-a) = f(-b)$ and $f((-a)b) = f((-b)a)$.
By~\eqref{2017a6-eq00}, we then get $f(b - a) = f(a - b)$.
By~\eqref{2017a6-eq4}, we get $2 f(a - b) = 2 f(0)$.
Since $G$ is $2$-torsion free, we then get $f(a - b) = f(0) \iff f(a - b + 1) = 0 \iff a = b$, as desired.
\end{proof}

We now assume that $f$ is injective and show that it has the desired form.
We take $a = \iota(f(0))$; we proved that $f(a^2) = f(1) = 0$ and so $a^2 = 1$.
Applying injectivity to~\eqref{2017a6-eq3} yields
\[ \iota(f(t)) = a(1 - t). \tag{5}\label{2017a6-eq5} \]
Comparing~\eqref{2017a6-eq00} with $(x, y) = (t, 0)$ and $(x, y) = (0, t)$ shows that $a \iota(f(t)) = \iota(f(t)) a$, or $a \cdot a(1 - t) = a(1 - t) \cdot a$, for any $t \in R$.
Replacing $t$ with $1 - at$ and applying~\eqref{2017a6-eq5} gives us $a \in Z(R)$.
Also due to~\eqref{2017a6-eq5},~\eqref{2017a6-eq00} reduces to
\[ f((1 - x)(1 - y)) + f(x + y) = f(xy). \tag{6}\label{2017a6-eq6} \]

\begin{claim}
Suppose that $G$ is $2$- and $3$-torsion free.
Then for any function $f : R \to G$ satisfying~\eqref{2017a6-eq6}, there exists a group homomorphism $\phi : R \to G$ such that $f(x) = \phi(1 - x)$ for all $x \in R$.
\end{claim}
\begin{proof}
Writing $g(x) = f(1 - x)$, we get
\[ g(x + y - xy) + g(1 - (x + y)) = g(1 - xy). \tag{7}\label{2017a6-eq7} \]
By plugging $y = 0$ into~\eqref{2017a6-eq7}, we get $g(x) + g(1 - x) = g(1)$ for all $x \in R$.
By plugging $y = 1$ instead, we get $g(1) + g(-x) = g(1 - x)$, which implies $g(x) + g(1) = g(x + 1)$ and $g(-x) = -g(x)$ for all $x \in R$.
In particular, $g(0) = 0$.

By induction on $n \in \Z$, we get $g(n) = n g(1)$ and $g(x + n) = g(x) + n g(1)$ for all $x \in R$.
By replacing $(x, y)$ with $(-x, -y)$ and using the equalities $g(-x) = -g(x)$ and $g(x + 1) = g(x) + 1$, one gets
\[ g(xy + x + y) = g(xy) + g(x + y). \]
By induction on $n \in \Z$, we then get that for any $x, y \in R$,
\[ g((x + n)(y + n)) = g(xy) + n g(x + y) + n^2 f(1). \tag{8}\label{2017a6-eq8} \]
Plugging $y = 0$ and replacing $x$ with $x - n$, we get $g(nx) = n g(x)$ for any $n \in \Z$ and $x \in R$.

Now replace $x$ with $2x$ in~\eqref{2017a6-eq8} and use $n = 2$.
Since $G$ is $2$-torsion-free, we get that for any $x, y \in R$,
\[ g((x + 1)(y + 2)) = g(xy) + g(2x + y) + 2 g(1). \]
By symmetry, we also get
\[ g((x + 2)(y + 1)) = g(xy) + g(x + 2y) + 2 g(1). \]
Combining the two equalities, we get
\[ g((x + 3)(y + 3)) = g(xy) + g(2x + y) + g(x + 2y) + 9 g(1). \]
Applying~\eqref{2017a6-eq8} yields $3 g(x + y) = g(2x + y) + g(x + 2y)$ for all $x, y \in R$.
By substituting $(x, y) = (2a - b, 2b - a)$, we get $3 g(a + b) = 3 (g(a) + g(b))$.
Since $G$ is $3$-torsion-free, we are done.
\end{proof}

Since left multiplication by $a$ is an invertible group homomorphism, we can instead write $f(x) = \phi(a(1 - x))$ for some group homomorphism $\phi : R \to G$.
Together with~\eqref{2017a6-eq5}, this implies that $\iota \circ \phi$ is the identity on $R$.
We are done.
