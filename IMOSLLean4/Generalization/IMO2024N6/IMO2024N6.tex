Given a ring $R$, we say that a function $f : R \to R$ is \textbf{good} if there exists $a, b, c \in R$ such that for any $r \in R$,
$$ (ar^2 + br + c)(f(r) - (ar^2 + br + c)) \neq 0. $$
We say that $R$ is \textbf{nice} if every polynomial over $R$ is good.
Find all nice finite commutative rings.



\subsection*{Answer}

We say that a ring $R$ is \textbf{boolean} if $r^2 = r$ for all $r \in R$.
Then a finite commutative ring is nice if and only if it is not boolean.



\subsection*{Solution}

For the case where $F$ is a finite field, we follow the \href{https://www.imo-official.org/problems/IMO2024SL.pdf}{official solution}.
Then we extend the result to finite commutative rings using our own arguments.

The constant $1$ polynomial is not good over a boolean ring, so boolean rings are not nice.
Thus it remains to show that any other finite commutative rings are nice.
We start by showing that any finite field of cardinality not equal to $2$ is nice.

\begin{theorem}
Let $F$ be a finite field.
Then $F$ is nice if and only if $|F| \neq 2$.
\end{theorem}
\begin{proof}
Solution 1 of the official solution solves the case where $F$ has characteristic not equal to $2$.
So now, assume that $F$ has characteristic $2$ and $|F| \neq 2$.
The rest of the proof adapts Solution 2 of the official solution.

Let $f : F \to F$ be an arbitrary function.
We start in similar fashion to Solution 1 of the official solution.
First, if $f$ does not attain some non-zero value, say $u$, then setting $a = b = 0$ and $c = u$ shows that $f$ is good.
On the other hand, if $f$ is not good, the function $g : F \to F$ defined by $g(x) = f(x)$ if $f(x) \neq 0$ and $g(x) = 1$ if $f(x) = 0$ is not good either.
Thus, we may assume $f(F) = F^{\times}$; in particular $f$ is not injective.
Second, if $f$ is good, then the function $x \mapsto f(ux + v)$ is good for any $u \in F^{\times}$ and $v \in F$.
Thus, we way assume in addition that $f(0) = f(1)$.

Consider the set $S = \{x^2 + x : x \in F\}$.
First fix an arbitrary element $x_0 \in F \setminus S$, and consider the choice $(a, b, c) = (a, a, -x_0 a)$ across all $a \in F^{\times}$.
Then to show that $f$ is good, it is enough to show that for any $x \in F$,
\[ f(x) \neq a(x^2 + x - x_0) \iff \frac{f(x)}{x^2 + x - x_0} \neq a. \]

Suppose that such $a$ does not exist for a given $x_0$.
Then the map $x \mapsto f(x)/(x^2 + x - x_0)$ is a surjection from $F$ to $F^{\times}$.
Since $f(0) = f(1)$, it is in fact a bijection from $F^{\times}$ to $F^{\times}$.
But so is $f$ by our initial assumption, so we get
\[ \prod_{x \in F^{\times}} \frac{f(x)}{x^2 + x - x_0} = \prod_{x \in F^{\times}} x = \prod_{x \in F^{\times}} f(x)
    \iff \prod_{x \in F^{\times}} (x^2 + x - x_0) = 1 \iff \prod_{x \in F} (x^2 + x - x_0) = -x_0 = x_0. \]

On the other hand, it can be shown that
\[ \prod_{x \in F} (x^2 + x - x_0) = 1, \]
    giving us a contradiction.
For example, we have
\[ \prod_{x \in F} (x^2 + x - x_0)^2
    = \prod_{x \in F} (x^4 + x^2 - x_0^2)
    = \prod_{x \in F} (x^2 + x - x_0^2)
    = \prod_{x \in F} ((x + x_0)^2 + (x + x_0) - x_0^2)
    = \prod_{x \in F} (x^2 + x - x_0), \]
    where the third equality holds since the map $x \mapsto x^2$ is bijective on $F$.
\end{proof}

It is easy to see that a ring that surjects into a nice ring is also nice.
Thus, if $R$ surjects into a finite field not isomorphic to $\F_2$, then $R$ is nice.
From now on, assume that for any maximal ideal $\km \subseteq R$, the quotient $R/\km$ is isomorphic to $\F_2$.

Our next step is the following lemma.

\begin{lemma}
Let $R$ be a ring.
Suppose that there exists a (maximal) ideal $\km$ of $R$ such that $R/\km \cong \F_2$ and $\km^2 \subsetneq \km$.
Then $R$ is nice.
\end{lemma}
\begin{proof}
Since $R/\km \cong \F_2$, $R$ is a disjoint union of $\km$ and $1 + \km$.
Now pick any element $\pi \in \km \setminus \km^2$.
Let $P \in R[X]$ be a polynomial; we want to show that $P$ is good.
\begin{itemize}
    \item   If $P(R) \subseteq \km$, then $(a, b, c) = (0, 0, 1)$ works.
    \item   If $P(R) \subseteq 1 + \km$, then $(a, b, c) = (0, 0, \pi)$ works.
    \item   If $P(\km) \subseteq \km$ and $P(1 + \km) \subseteq 1 + \km$, then $(a, b, c) = (1, 0, \pi - 1)$ works.
    \item   If $P(\km) \subseteq 1 + \km$ and $P(1 + \km) \subseteq \km$, then $(a, b, c) = (1, 0, \pi)$ works.
\end{itemize}
To see why the choices work, note that they are crafted so that $P(x) \not\equiv ax^2 + bx + c \pmod{\km}$ and $ax^2 + bx + c \notin \km^2$ for any $x \in R$. 
\end{proof}

We are now ready for the final step.
Suppose that $R$ is a finite commutative ring that is not nice.
Then for every maximal ideal $\km$ of $R$, $R/\km$ is isomorphic to $\F_2$.
By the above lemma, we must have $\km^2 = \km$ for all maximal ideals $\km$.

Let $J$ denote the Jacobson radical of $R$.
Since $R$ is finite, thus artinian, $J$ is also equal to the nilradical of $R$ and $J^n = 0$ for some $n \geq 1$.
But also $J$ is equal to the product of all maximal ideals of $R$.
Since the maximal ideals $\km$ satisfy $\km^2 = \km$, we get $J^2 = J$ and so $J^n = J$ for all $n \geq 1$.
This forces $J = 0$, so $R$ is reduced.

Since $R$ is a reduced and finite, $R$ is isomorphic to the product of $R/\km$ across all maximal ideals $\km \leq R$.
Recall that $R/\km \cong \F_2$ for each maximal ideal $\km$; thus $R$ is isomorphic to the product of copies of $\F_2$.
This forces $R$ to be boolean, as desired.
