Given a ring $R$, we say that a function $f : R \to R$ is \textbf{good} if there exists $a, b, c \in R$ such that for any $r \in R$,
$$ (ar^2 + br + c)(f(r) - (ar^2 + br + c)) \neq 0. $$
We say that $R$ is \textbf{nice} if every polynomial over $R$ is good.
Find all nice finite commutative rings.



\subsection*{Progress}

A finite field is nice if and only if it has cardinality not equal to $2$.



\subsection*{Solution}

This solution follows the outline of Solution 2 of the \href{https://www.imo-official.org/problems/IMO2024SL.pdf}{official solution}.
As shown in the official solution, the only prime number $p$ such that $\F_p$ is not nice is $p = 2$.
Thus we just have to prove that finite fields of characteristic $2$ other than $\F_2$ are nice.

Let $F$ be a finite field of characteristic $2$ with $|F| \neq 2$.
Let $f : F \to F$ be an arbitrary function.
We start in similar fashion to Solution 1 of the official solution.

First, if $f$ does not attain some non-zero value, say $u$, then setting $a = b = 0$ and $c = u$ shows that $f$ is good.
On the other hand, if $f$ is not good, the function $g : F \to F$ defined by $g(x) = f(x)$ if $f(x) \neq 0$ and $g(x) = 1$ if $f(x) = 0$ is not good either.
Thus, we may assume $f(F) = F^{\times}$; in particular $f$ is not injective.
Second, if $f$ is good, then the function $x \mapsto f(ux + v)$ is good for any $u \in F^{\times}$ and $v \in F$.
Thus, we way assume in addition that $f(0) = f(1)$.

Consider the set $S = \{x^2 + x : x \in F\}$.
First fix an arbitrary element $x_0 \in F \setminus S$, and consider the choice $(a, b, c) = (a, a, -x_0 a)$ across all $a \in F^{\times}$.
Then to show that $f$ is good, it is enough to show that for any $x \in F$,
\[ f(x) \neq a(x^2 + x - x_0) \iff \frac{f(x)}{x^2 + x - x_0} \neq a. \]

Suppose that such $a$ does not exist for a given $x_0$.
Then the map $x \mapsto f(x)/(x^2 + x - x_0)$ is a surjection from $F$ to $F^{\times}$.
Since $f(0) = f(1)$, it is in fact a bijection from $F^{\times}$ to $F^{\times}$.
But so is $f$ by our initial assumption, so we get
\[ \prod_{x \in F^{\times}} \frac{f(x)}{x^2 + x - x_0} = \prod_{x \in F^{\times}} x = \prod_{x \in F^{\times}} f(x)
    \iff \prod_{x \in F^{\times}} (x^2 + x - x_0) = 1 \iff \prod_{x \in F} (x^2 + x - x_0) = -x_0. \]

On the other hand, it can be shown that
\[ \prod_{x \in F} (x^2 + x - x_0) = 1. \]
For example, we have
\[ \prod_{x \in F} (x^2 + x - x_0)^2
    = \prod_{x \in F} (x^4 + x^2 - x_0^2)
    = \prod_{x \in F} (x^2 + x - x_0^2)
    = \prod_{x \in F} ((x + x_0)^2 + (x + x_0) - x_0^2)
    = \prod_{x \in F} (x^2 + x - x_0), \]
    where the third equality holds since the map $x \mapsto x^2$ is bijective on $F$.
    