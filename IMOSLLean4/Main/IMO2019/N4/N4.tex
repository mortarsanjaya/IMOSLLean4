Fix some $C \geq 0$.
Find all functions $f : \N \to \N$ such that $a + f(b) \mid a^2 + b f(a)$ for any $a, b \in \N$ satisfying $a + b > C$.



\subsection*{Answer}

$n \mapsto kn$ for some $k \in \N$, regardless of $C$.



\subsection*{Solution}

Official solution: \url{https://www.imo-official.org/problems/IMO2019SL.pdf}

We modify Solution 1 of the official solution to work in our setting.
We also avoid implicitly using pigeonhole principle.

First, plugging $a = 1$ gives $1 + f(b) \mid 1 + b f(1)$, which implies
\[ f(b) \leq b f(1) \quad \forall b > C. \tag{1}\label{2019n4-eq1} \]
Next, for any $b > 0$, we show that $b \mid f(b)^2$.
Indeed, there exists $n \in \N$ such that $nb > C + f(b)$.
Then plugging $a = nb - f(b)$ yields $b \mid a^2$ and thus $b \mid f(b)^2$ since $b \mid a + f(b)$.
In particular, for any $p$ prime with $p > C$, we have $p \mid f(p)$.
Thus, by~\eqref{2019n4-eq1}, we can write $f(p) = kp$ for some $k \leq f(1)$.
Now we prove the following claim.

\begin{claim}
Let $x \in \N$ and choose a prime $p > \max\{C, f(x), x f(1), x^2\}$.
Then we can write $f(p) = kp$ for some $k \in \N$, and we also have $f(x) = kx$.
\end{claim}
\begin{proof}
If $f(x) = 0$, plugging $(a, b) = (x, p)$ gives $x + kp \mid x^2$.
If $k = 0$, then $f(x) = 0 = kx$ and we are done.
If $k > 0$, then $x + kp > x^2$ forces $x = 0$ and again, we are done.

Now assume that $f(x) > 0$.
Plugging $(a, b) = (p, x)$ yields $p + f(x) \mid p^2 + x f(p) = p (p + kx)$.
Since $0 < f(x) < p$, we know that $p$ and $p + f(x)$ are coprime, so $p + f(x) \mid p + kx$.
Since $p > C$, we know that $k \leq f(1)$, so $p + kx \leq p + x f(1) < 2p$.
Then this means that $p + f(x) = p + kx$, and thus $f(x) = kx$.
\end{proof}

For simplicity, write $B_x = \max\{C, f(x), x f(1), x^2\}$.
Choose an arbitrary prime $p > C$, and write $f(p) = kp$ for some $k \leq f(1)$.
We prove that $f(n) = kn$ for any $n \in \N$.
Indeed, choose a prime $q > \max\{B_n, B_p\}$.
Then by the claim, we can write $f(q) = mq$ for some $m \in \N$ and that $kp = f(p) = mp$.
The latter yields $k = m$, and applying the claim to $n$ yields $f(n) = mn = kn$, as desired.



\subsection*{Extra notes}

The original problem can actually be deduced from this alternative problem.
Indeed, fix some $g : \N^+ \to \N^+$ satisfying the original condition.
Extend $g$ to $f : \N \to \N$ by defining $f(0) = 0$.
Then one could verify that $f$ satisfies the above condition.
Thus, by our solution above, $f = n \mapsto kn$ for some $k \in \N$.
The same holds for $g$, and $k > 0$ since $f \neq 0$.
