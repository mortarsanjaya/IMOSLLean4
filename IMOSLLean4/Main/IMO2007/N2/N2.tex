Fix an integer $b > 0$ and $n \geq 0$.
Suppose that for each positive integer $k$, there exists an integer $a$ such that $k \mid b - a^n$.
Prove that $b = A^n$ for some integer $A$.



\subsection*{Solution}

Official solution: \url{https://www.imo-official.org/problems/IMO2007SL.pdf}

As in the official solution, we only need the case $k = b^2$.
However, there is a very short solution as follows.

Since $b^2 \mid b - c^n$ for some integer $c$, we can write $c^n = b - ab^2 = (1 - ab) b$ for some integer $a$.
But $1 - ab$ and $b$ are coprime, so indeed $b$ must be an $n^{\text{th}}$ power.



\subsection*{Extra notes}

As noted in the official solution, the problem is false if we only require $b$ to be $n^{\text{th}}$ power modulo primes.
An example is $16 = 2^4$, which is an $8^{\text{th}}$ power modulo any prime.
However, using algebraic number theory, it can be shown that this constitutes pretty much all the counter-examples.
That is, if $b$ is $n^{\text{th}}$ power modulo all primes but not an $n^{\text{th}}$ power, then $n = 8k$ and $b = (2c^2)^{4k}$ for some integer $c$.
We will not prove this or formalize its proof.
