Find all functions $f : \N \to \N$ such that, for any $n \in \N$,
\[ f(f(f(n))) = f(n + 1) + 1. \tag{*}\label{2013a5-eq0} \]



\subsection*{Answer}

The function $n \mapsto n + 1$ and the function $\phi : \N \to \N$ defined below:
\[ \phi(n) \mapsto \begin{cases}
    n + 5, & \text{if } n \equiv 1 \pmod{4}, \\
    n - 3, & \text{if } n \equiv 3 \pmod{4}, \\
    n + 1, & \text{if } n \equiv 0 \pmod{2}.
\end{cases} \]
Note that $\phi$ can be defined recursively via $\phi(0) = 1$, $\phi(1) = 6$, $\phi(2) = 3$, $\phi(3) = 0$, and $\phi(n + 4) = \phi(n) + 4$ for all $n \in \N$.



\subsection*{Solution}

Official solution: \url{https://www.imo-official.org/problems/IMO2013SL.pdf}

We follow Solution 1 of the official solution.
It is easy to check that the two answers indeed work.
The rest of the solution is for proving that they are the only functions satisfying~\eqref{2013a5-eq0}.

First of all, we prove that $f^4(n + 1) = f^4(n) + 1$ for all $n \in \N$.
Indeed, we have
\[ f^4(n + 1) = f^3(f(n + 1)) = f(f(n + 1) + 1) + 1 = f(f^3(n)) + 1 = f^4(n) + 1. \]
Then by induction, we get $f^4(n) = n + f^4(0)$ for all $n \in \N$.
This also means that $f$ is injective.

The next observation is that the set $\N \setminus f^4(\N)$ is finite.
Indeed, it is exactly the set of non-negative integers less thatn $f^4(0)$.
In particular, $\N \setminus f(\N)$ is finite as well.

For each non-negative integer $k$, let $S_k = f^k(\N) \setminus f^{k + 1}(\N)$.
Since $f$ is injective, it induces a bijection between $S_k$ and $S_{k + 1}$.
Thus the $S_k$s have the same cardinality as $S_0 = \N \setminus f(\N)$, which is finite.
Also, since $f$ is injective, the $S_k$s are pairwise disjoint.
Thus, on one hand, we have $\#(\N \setminus f^3(\N)) = 3 \cdot \#(\N \setminus f(\N))$.
On the other hand, by~\eqref{2013a5-eq0} and injectivity of $f$, we can see that
\[ \N \setminus f^3(\N) \subseteq \{0, f(0) + 1\} \cup \{x + 1 : x \notin f(\N)\}. \]
This set in the right hand side has size at most $2 + \#(\N \setminus f(\N))$.
Thus $\#(\N \setminus f(\N)) \leq 1$, and equality holds since $f$ cannot be surjective.
In particular, we obtain
\[ f^4(0) = \#(\N \setminus f^4(\N)) = 4 \cdot \#(\N \setminus f(\N)) = 4, \]
    and for all $n \in \N$,
\[ f(n + 4) = f(n + f^4(0)) = f^5(n) = f(n) + f^4(0) = f(n) + 4. \tag{1}\label{2013a5-eq1} \]
Based on the answers, it now suffices to find the values of $f(0), f(1), f(2), f(3)$.

Recall that $\#(\N \setminus f(\N)) = 1$.
Thus there exists a unique non-negative integer not attained by ç$f$, say $a$.
Then we get
\[ \{a, f(a), f^2(a)\} \subseteq \{0, f(0) + 1, a + 1\}. \]
Note that~\eqref{2013a5-eq1} yields $f(n), f^2(n) \neq n$ for any $n \in \N$.
Since $a \neq a + 1$ is clear, we have two cases:

\begin{itemize}

    \item
    Case 1: $a = 0$.

    Then $f(0) \neq 0$ and $f(0) \neq f(0) + 1$, so $f(0) = 0 + 1 = 1$.
    Similarly, $f^2(0) \neq 0$ and $f^2(0) = f(1) \neq 1$, so $f^2(0) = f(0) + 1 \implies f(1) = 2$.
    By~\eqref{2013a5-eq0}, we then get $f^3(0) = f(1) + 1 \implies f(2) = 3$.
    Then we get $f(3) = f^4(0) = 4$, so $f(n) = n + 1$ for $n < 4$.
    Finally,~\eqref{2013a5-eq1} yields $f(n) = n + 1$ for all $n \in \N$.

    \item
    Case 2: $a = f(0) + 1$.

    Then $f(a), f^2(a) \in \{0, a + 1\}$.
    If $f(a) = 0$, then $f(0) = f^2(a) = a + 1 = f(0) + 2$; a contradiction.
    So $f(a) = a + 1$ and $f(a + 1) = f^2(a) = 0$.
    Then~\eqref{2013a5-eq0} yields $f^3(a) = f(a + 1) + 1 \implies f(0) = 1$.
    We get $a = 2$, and so $f(2) = 3$, $f(3) = 0$, and $f(0) = 1$.
    Using~\eqref{2013a5-eq0} again yields $f(1) = 6$.
    Finally,~\eqref{2013a5-eq1} yields the second answer.

\end{itemize}
