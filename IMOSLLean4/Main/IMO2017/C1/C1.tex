For any $(a, b, w, h) \in \N^4$, let $R(a, b, w, h)$ denote the set of pairs $(x, y) \in \N^2$ with $a \leq x < a + w$ and $b \leq y < b + h$.
This is a "lattice" rectangle, which can be viewed as the intersection of the real rectangle $[a, a + w) \times [b, b + h)$ with the lattice $\Z^2$.

Fix $m, n \in \N$ odd.
Consider a partition of $R(0, 0, m, n)$ into $k \geq 1$ rectangles, say $R(a_i, b_i, w_i, h_i)$ for $i = 1, 2, \ldots, k$.
Prove that, for at least one of the rectangles $R(a_i, b_i, w_i, h_i)$, the distance of its sides to the sides of $R(0, 0, m, n)$ are all odd or all even.
That is, prove that there exists at least one index $i$ such that $w_i$ and $h_i$ are odd and $a_i \equiv b_i \pmod{2}$.



\subsection*{Solution}

Official solution: \url{http://www.imo-official.org/problems/IMO2017SL.pdf}

The solution we describe below is just a copy of the official solution.
Some rewording are made for the purpose of easier transition into Lean codes.
In fact, this is done for the problem statement as well.

We will simplify by using short notations for quadruples.
For each $i \leq k$, denote $q_i = (a_i, b_i, w_i, h_i)$.
We put some weight on the points in $\N^2$, with $(x, y)$ having weight $(-1)^{x + y}$.
The weight of a finite subset $S \subseteq \N^2$ is the sum of the weight of the points in $S$.
For each quadruple $q \in \N^4$, denote by $f(q)$ the weight of $R(q)$, i.e.,
\[ f(q) = \sum_{(x, y) \in R(q)} (-1)^{x + y}. \]

\begin{claim}
Let $q = (a, b, w, h)$.
If either $w$ or $h$ is even, then $f(q) = 0$.
If $w$ and $h$ are odd, then $f(q) = 1$ if $a \equiv b \pmod{2}$ and $f(q) = -1$ otherwise.
Equivalently, $f(q) = (-1)^{a + b}$.
\end{claim}
\begin{proof}
The claim follows from the equality
\[ f(q) = \frac{(-1)^{a + w} - (-1)^a}{2} \cdot \frac{(-1)^{b + h} - (-1)^b}{2} = (-1)^{a + b}\frac{((-1)^w - 1)((-1)^h - 1)}{4}. \]
\end{proof}

Now, since $R(q)$ is a disjoint union of the $R(q_i)$s, we have $f(q) = \sum_{i \leq k} f(q_i)$.
But the claim yields $f(q) = 1$ and $f(q_i) \in \{-1, 0, 1\}$ for each $i \leq k$.
Thus $f(q_i) = 1$ for some $i \leq k$.
This choice of $q_i$ works by the claim.



\subsection*{Implementation details}

Instead of using the standard predicate \texttt{Odd} and \texttt{Even}, we use the boolean function \texttt{Nat.bodd}.
Taking an input $n \in \N$, it returns \texttt{true} if $n$ is odd and \texttt{false} if $n$ is even.
This reduces import insignificantly, but it allows more direct computations.
