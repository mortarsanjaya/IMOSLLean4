Let $R$ be a ring and $S$ be a domain.
Find all functions $f : R \to S$ such that, for any $x, y \in R$,
\[ f(xy + 1) = f(x) f(y) + f(x + y). \tag{*}\label{2012a5-eq0} \]

\textbf{Warning.}
The original problem uses the above functional equation with $R = S = \R$ and an extra condition, $f(-1) \neq 0$.
This problem is an extreme buffed version of the original problem.
The difficulty level of this problem is far beyond the level of IMO.
Proceed with caution; the solution might be very hard to understand.









\subsection*{Answer}

Before proceeding, note the following.
Given a good function $f : R \to S$ and homomorphisms $\phi : R' \to R$ and $\iota : S \to S'$, it is easy to check that $\iota \circ f \circ \phi$ is good.
The functions we will list below are the "initial" functions.
All good functions are of form $\phi \circ f \circ \iota$ where $f$ is a good function on the list.
Here is the list of initial functions:

\begin{itemize}

    \item
    The zero function.
    
    \item
    The function $x \in R \mapsto x - 1$ on a ring $R$.

    \item
    The function $x \in R \mapsto x^2 - 1 \in R_2$, where $R$ is a commutative ring and $R_2$ is the subring of $R$ generated by squares.
    
    \item
    One of the six functions below:
    \begin{align*}
        \F_2 &\to \Z    & \F_3 &\to \Z    & \F_3 &\to \Z    & \Z/4\Z &\to \Z  & \F_2[X]/\langle X^2 \rangle &\to \Z   & \F_4 &\to \Z[\varphi] \\
        0 &\mapsto -1   & 0 &\mapsto -1   & 0 &\mapsto -1   & 0 &\mapsto -1   & 0 &\mapsto -1                         & 0 &\mapsto -1 \\
        1 &\mapsto 0    & 1 &\mapsto 0    & 1 &\mapsto 0    & 1 &\mapsto 0    & 1 &\mapsto 0                          & 1 &\mapsto 0  \\
                       && 2 &\mapsto 1    & 2 &\mapsto 0    & 2 &\mapsto 1    & X &\mapsto 1                          & X &\mapsto \varphi  \\
                                                       &&&&&& 3 &\mapsto 0    & X + 1 &\mapsto 0                      & X + 1 &\mapsto 1 - \varphi
    \end{align*}
    For the last function, $\varphi$ is the golden ratio $\frac{1 + \sqrt{5}}{2}$.

\end{itemize}









\subsection*{Solution}

It is easy to check that the zero function and $x \mapsto x - 1$ are good.
To check the third function, it suffices to check that $x \mapsto x^2 - 1$ on a commutative ring is good, which can be bashed.
Finally, the six functions in the above table have finite input ring, so again we can bash to verify that they are all good functions.
Now we focus on proving that there are no other good functions.

We say that a good function $f : R \to S$ is \emph{non-trivial good} if $f(1) = 0$ and $f(0) = -1$.
We say that $f$ is \emph{reduced good} if $f$ is non-trivial and $f$ has no non-zero periodic element.
That is, $f(x + c) = f(x + d)$ for all $x \in R$ implies $c = d$.

We start with some easy observations.
Plugging $x = y = 1$ yields $f(1) = 0$.
Then plugging $y = 0$ yields either $f \equiv 0$ and $f(0) = -1$.
That is, if $f$ is not the zero function, then $f$ is non-trivial good.
We now continue with more observations.




\subsubsection*{A solver for $f(x + 1) = f(x) + 1$.}

We now solve the problem when $f(x + 1) = f(x) + 1$ for all $x \in R$.
The result is as follows:

\begin{lemma}\label{2012a5-linear-solver}
Let $f : R \to S$ be a non-trivial good map such that $f(x + 1) = f(x) + 1$ for all $x \in R$.
Then $f + 1$ is a ring homomorphism.
\end{lemma}
\begin{proof}
First we reduce the lemma to showing that $f + 1$ is additive.
Equivalently, $f(x + y) = f(x) + f(y) + 1$ for all $x, y \in R$.
Indeed, if $f + 1$ is additive, then for any $x, y \in R$,
\[ f(xy) + 1 = f(xy + 1) = f(x) f(y) + f(x + y) = f(x) f(y) + f(x) + f(y) + 1 = (f(x) + 1)(f(y) + 1). \]
Thus $f + 1$ is multiplicative.
Since $f(1) + 1 = 1$, this would prove that $f + 1$ is a ring homomorphism.
Before proceeding, note that for any $x, y \in R$,
\[ f(xy) = f(xy + 1) - 1 = f(x) f(y) + f(x + y) - 1. \]

Now fix $x, y \in R$; we show that $f(x + y) = f(x) + f(y) + 1$.
The trick is to write $f(x(x + 1) y + 1)$ in two ways.
Let $a = f(x)$, $b = f(y)$, and $c = f(x + y)$, and note again that $f(x + 1) = f(x) + 1$ for all $x \in R$.
We get
\begin{align*}
    a f((x + 1) y) + f(x + (x + 1) y)
        &= (a + 1) f(xy) + f(x + 1 + xy) \\
        a((a + 1) b + c) + (a + 1)(b + 1) + c
            &= (a + 1) (ab + c - 1) + a(b + 1) + c + 1
\end{align*}
    which simplifies to $c = a + b + 1$ after some heavy algebraic manipulation.
It is the desired equality.
\end{proof}




%% ...
