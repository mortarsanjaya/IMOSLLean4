Let $R$ be a ring and $S$ be a domain.
Find all functions $f : R \to S$ such that, for any $x, y \in R$,
\[ f(xy + 1) = f(x) f(y) + f(x + y). \tag{*}\label{2012a5-eq0} \]

\textbf{Warning.}
The original problem uses the above functional equation with $R = S = \R$ and an extra condition, $f(-1) \neq 0$.
This problem is an extreme buffed version of the original problem.
The difficulty level of this problem is far beyond the level of IMO.
Proceed with caution; the solution might be very hard to understand.









\subsection*{Answer}

All good functions are as follows:

\begin{itemize}

    \item
    The zero function $0 : R \to S$ is good.
    
    \item
    For any ring homomorphism $\phi : R \to S$, the function $x \mapsto \phi(x) - 1$ is good.

    \item
    Let $R^{ab} = R/[R, R]$, where $[R, R]$ is the commutator ideal of $R$.
    Let $R_2$ be the subring of $R^{ab}$ generated by squares.
    Consider the canonical map $\rho : R \to R_2$ given by taking quotient to $R^{ab}$ and then squaring.
    Then $x \mapsto \phi(\rho(x)) - 1$ is good for any ring homomorphism $\phi : R_2 \to S$.

    \item
    Let $g : R' \to S'$ be one of the six following functions.
    Let $\phi : R \to R'$ and $\iota : S' \to S$ be ring homomorphisms.
    Then $\phi \circ g \circ \iota$ is good.
    \begin{align*}
        \F_2 &\to \Z    & \F_3 &\to \Z    & \F_3 &\to \Z    & \Z/4\Z &\to \Z  & \F_2[X]/\langle X^2 \rangle &\to \Z   & \F_4 &\to \Z[\varphi] \\
        0 &\mapsto -1   & 0 &\mapsto -1   & 0 &\mapsto -1   & 0 &\mapsto -1   & 0 &\mapsto -1                         & 0 &\mapsto -1 \\
        1 &\mapsto 0    & 1 &\mapsto 0    & 1 &\mapsto 0    & 1 &\mapsto 0    & 1 &\mapsto 0                          & 1 &\mapsto 0  \\
                       && 2 &\mapsto 1    & 2 &\mapsto 0    & 2 &\mapsto 1    & X &\mapsto 1                          & X &\mapsto \varphi  \\
                                                       &&&&&& 3 &\mapsto 0    & X + 1 &\mapsto 0                      & X + 1 &\mapsto 1 - \varphi
    \end{align*}
    For the last function, $\varphi$ is the golden ratio $\frac{1 + \sqrt{5}}{2}$.

\end{itemize}









\subsection*{Solution}

Given a good function $f : R \to S$ and homomorphisms $\phi : R' \to R$ and $\iota : S \to S'$, it is easy to check that $\iota \circ f \circ \phi$ is good.
We will use this fact to check the functions given in the answer section, but we will use it later on as well, at least implicitly.

It is easy to check that the zero function and $x \mapsto x - 1$ are good.
The six functions in the above list have finite input ring, so one can check by hand that they are all good functions.
To check for the third function involving $R_2 \subseteq R^{ab} = R/[R, R]$, by the previous paragraph, it suffices to check that $x \mapsto x^2 - 1$ is good on $R^{ab}$.
But this can be done by hand since $R^{ab}$ is abelian.
This verifies that all functions given in the answer section are good.
Now we focus on proving that there are no other good functions.

We say that a good function $f : R \to S$ is \emph{non-trivial good} if $f(1) = 0$ and $f(0) = -1$.
We say that $f$ is \emph{reduced good} if $f$ is non-trivial and $f$ has no non-zero periodic element.
That is, $f(x + c) = f(x + d)$ for all $x \in R$ implies $c = d$.

We start with some easy observations.
Plugging $x = y = 1$ yields $f(1) = 0$.
Then plugging $y = 0$ yields either $f \equiv 0$ and $f(0) = -1$.
That is, if $f$ is not the zero function, then $f$ is non-trivial good.




\subsubsection*{A solver for $f(x + 1) = f(x) + 1$.}

We now solve the problem when $f(x + 1) = f(x) + 1$ for all $x \in R$.
The result is as follows:

\begin{lemma}\label{2012a5-linear-solver}
Let $f : R \to S$ be a non-trivial good map such that $f(x + 1) = f(x) + 1$ for all $x \in R$.
Then $f + 1$ is a ring homomorphism.
\end{lemma}
\begin{proof}
First we reduce the lemma to showing that $f + 1$ is additive.
Equivalently, $f(x + y) = f(x) + f(y) + 1$ for all $x, y \in R$.
Indeed, if $f + 1$ is additive, then for any $x, y \in R$,
\[ f(xy) + 1 = f(xy + 1) = f(x) f(y) + f(x + y) = f(x) f(y) + f(x) + f(y) + 1 = (f(x) + 1)(f(y) + 1). \]
Thus $f + 1$ is multiplicative.
Since $f(1) + 1 = 1$, this would prove that $f + 1$ is a ring homomorphism.
Before proceeding, note that for any $x, y \in R$,
\[ f(xy) = f(xy + 1) - 1 = f(x) f(y) + f(x + y) - 1. \]

Now fix $x, y \in R$; we show that $f(x + y) = f(x) + f(y) + 1$.
The trick is to write $f(x(x + 1) y + 1)$ in two ways.
Let $a = f(x)$, $b = f(y)$, and $c = f(x + y)$, and note again that $f(x + 1) = f(x) + 1$ for all $x \in R$.
We get
\begin{align*}
    a f((x + 1) y) + f(x + (x + 1) y)
        &= (a + 1) f(xy) + f(x + 1 + xy) \\
        a((a + 1) b + c) + (a + 1)(b + 1) + c
            &= (a + 1) (ab + c - 1) + a(b + 1) + c + 1
\end{align*}
    which simplifies to $c = a + b + 1$ after some heavy algebraic manipulation.
It is the desired equality.
\end{proof}




%% ...
